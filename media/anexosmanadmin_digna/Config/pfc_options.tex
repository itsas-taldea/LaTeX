%
% Paquetes que pueden serte de utilidad (rec = recomendado, opc = opcional)
%
\usepackage{fancyhdr}          % (rec)  permite cambiar varios par�metros de las cabeceras y pi�s de p�gina
\usepackage{fancyvrb}		% (rec) permite cambiar parámetros del texto
% literal (entorno verbatim)
\usepackage{courier}           % (opc)  usa esta fuente por defecto
\usepackage[spanish]{babel}   % (rec)  da soporte para castellano a LaTeX
\usepackage[utf8]{inputenc}    % (rec) soporte para tildes
%\usepackage{setspace}          % (opc)  permite cambiar el espaciado entre l�neas
%\usepackage{longtable}         % (opc)  permite que las tablas ocupen varias p�ginas
%\usepackage{lscape}            % (opc)  permite el uso del comando \landscape, para poner algo apaisado
\usepackage{color}             % (opc)  varios comandos relativos al color (como
% \color)
% \usepackage{colortbl}           % (opc) permite usar color en las tablas
% \usepackage{rotating}          % (opc)  permite rotar PSs y EPSs
% \usepackage{textcomp}          % (opc)  permite incluir el s�mbolo del euro,
% con \texteuro
%\usepackage{minitoc}           % (opc)  permite incluir ToCs (�ndice de materias) para cada cap�tulo
%\usepackage{epsf}              % (opc)  permite ciertas manipulaciones a EPSs
\usepackage[absolute]{textpos} % (rec)  permite posicionado arbitrario de texto (necesario para la portada)
%\usepackage{srcltx}            % (opc)  permite pasar del .dvi al .tex
% \usepackage{marvosym}
\usepackage[breaklinks=true, colorlinks]{hyperref} % (opc) hace que las
% referencias cruzadas se conviertan en enlaces
% \usepackage{titleref}           % (opc) permite hacer referencia a otra
% sección mostrando su título
\usepackage{graphicx}
% \usepackage{multirow}
% \usepackage{setspace}
\usepackage{url}     % Enlaces web
% \usepackage{tabularx} %Ajuste automático del ancho de columnas
% \usepackage{longtable} %Tablas de más de una página, por ejemplo para los
% acrónimos
\usepackage{listings} %Mostrar código en el texto, por ejemplo comandos de consola
\usepackage{cmap}   %Hacer que el pdf sea searchable

\pretolerance=10000 %Para evitar que se corten las palabras
\tolerance=10000  %Para evitar que se corten las palabras
\usepackage[scaled=0.92]{helvet}
\renewcommand{\sfdefault}{phv} %para cambiar el tipo de fuente por defecto
%Comando para configurar el documento en Arial
\sffamily %trabaja con helvetica en el cuerpo del documento
\renewcommand{\rmfamily}{phv} % permite trabajar con helvetica los títulos
\renewcommand{\familydefault}{phv} % permite trabajar con letra helvetica los títulos

\definecolor{gray97}{gray}{.97}
\definecolor{gray75}{gray}{.75}
\definecolor{gray45}{gray}{.45}
 
\lstset{ frame=Ltb,
     framerule=0pt,
     aboveskip=0.5cm,
     framextopmargin=3pt,
     framexbottommargin=3pt,
     framexleftmargin=0.4cm,
     framesep=0pt,
     rulesep=.4pt,
     backgroundcolor=\color{gray97},
     rulesepcolor=\color{black},
	inputencoding=utf8,
	extendedchars=\true,
     %
     stringstyle=\ttfamily,
     showstringspaces = false,
     basicstyle=\small\ttfamily,
     commentstyle=\color{gray45},
     keywordstyle=\bfseries,
     %
     numbers=left,
     numbersep=15pt,
     numberstyle=\tiny,
     numberfirstline = false,
     breaklines=true,
     escapeinside=+ +,
   }
 
% minimizar fragmentado de listados
\lstnewenvironment{listing}[1][]
   {\lstset{#1}\pagebreak[0]}{\pagebreak[0]}
 
\lstdefinestyle{consola}
   {basicstyle=\scriptsize\bf\ttfamily,
    backgroundcolor=\color{gray75},
   }

\lstdefinestyle{archivo}
   {basicstyle=\scriptsize\ttfamily,
    backgroundcolor=\color{gray97},
   }

%\usepackage{urlbst} %Soporte para url en bibtex
%
% Settings para los m�rgenes. Descomenta y modifica si sabes lo que haces. N�tese
% que a los valores dados se les a�ade una pulgada extra. Los valores dados son los
% predeterminados para papel A4 y el estilo itsas_pfc.cls.
%
%\setlength{\oddsidemargin}{10pt}     % m�rgen izquierdo para p�ginas impares (izquierda)
%\setlength{\evensidemargin}{52pt}    % m�rgen izquierdo para p�ginas pares (derecha)
%\setlength{\textwidth}{390pt}        % anchura del cuerpo de texto

%
% Recomendado para mejorar la colocaci�n autom�tica de las figuras.
% (tomado de http://dcwww.camp.dtu.dk/~schiotz/comp/LatexTips/LatexTips.html#captfont)
%
\renewcommand{\topfraction}{0.85}
\renewcommand{\textfraction}{0.1}
\renewcommand{\floatpagefraction}{0.75}

%
% Espacio entre el borde superior de la p�gina y donde comienza el texto (ah� van las
% cabeceras). LaTeX se queja si usamos el paquete fanchyhdr y headheight es menor de 15pt
%
\headheight 16pt

%
% Para el paquete textpos (usado para la portada)
%
\setlength{\TPHorizModule}{\paperwidth}
\setlength{\TPVertModule}{\paperheight}
\newcommand{\tb}[4]{\begin{textblock}{#1}[0.5,0.5](#2,#3)\begin{center}#4\end{center}\end{textblock}}

\newcommand{\respuesta}[1]{\setlength{\parindent}{0pt}\colorbox[gray]{0.87}{\scriptsize\texttt{\BUseVerbatim{#1}}}\setlength{\parindent}{1cm}}

%
% Aqu� puedes definir tus comandos.
% 
% \newcommand{cmd}[args]{def}
%
% cmd  = el comando a definir (p.e. \cadena)
% args = el n�mero de argumentos
% def  = la definici�n, sustituyendo #1, #2... por el primer, segundo... argumento
%
% Por ejemplo:
%
% \newcommand{\agua}[1]{H\ensuremath{_#1}O}
%
% Cada vez que escribamos "\agua{33}", en el output saldr�: "H33O" (con el 33 como sub�ndice)
%

%\newcommand{\algo}{algo}
% 
% \newcommand{\todolist}[1]{
%  \marginpar{
%   \fbox{
%    \begin{minipage}{3cm}
%     \tiny{
%      \begin{list}{$\bullet$}
%       {
%       \textsc{\textbf{To do: }}\vspace*{-0.2cm}
%       \setlength\labelwidth{0.1cm}
%       \setlength\itemindent{0cm}
%       \setlength\leftmargin{0.05cm}
%       \setlength\parsep{0cm}
%       \setlength\itemsep{0.05cm}
%       }
%       #1
%      \end{list}
%     }
%    \end{minipage}
%   }
%  }
% }

%
% Aqu� puedes instruir a LaTeX de por d�nde cortar las palabras que �l autom�ticamente
% no sepa. P.e., para cortar "gnomonly" solo por donde se se�ala con guiones (-).
%
% \hyphenation{gno-mon-ly} 
 
%
% Que las primeras p�ginas sean numeradas con n�meros romanos.
% M�s adelante se cambiar� de nuevo a ar�bicos.
%
\pagenumbering{Roman}
