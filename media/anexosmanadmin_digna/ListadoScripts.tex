\section{LISTADO DE SCRIPTS}
\label{scripts}

\subsection{Sanity.pl}
\label{script-sanity}
\begin{lstlisting}[style=archivo]
#!/usr/bin/perl -w
#sanity.pl
use File::Basename;
use Getopt::Std;

# check for unicodedecoder
my $unidec = 0;
eval { require Text::Unidecode; };
unless ($@) {
  $unidec = 1;
  Text::Unidecode ->import();
}else{
}

# check commandline params
my %OPTS;
getopts('le',\%OPTS);

# Function prototypes:
sub readFiles($);
sub renameFile($$);
sub help();

#################################################################
# rename a given File
sub renameFile($$){

	(my $path,$file) = @_;
	my $newfile = $file;

	#remove chars below 32
	$newfile =~ s/[\x00-\x1f]/_/g;

	#urldecode:
	$newfile =~ s/%([0-9A-Fa-f][0-9A-Fa-f])/chr(hex($1))/ge;

  #fix broken unicode chars for german umlauts
  $newfile =~ s/\303\204/Ae/g;
  $newfile =~ s/\303\226/Oe/g;
  $newfile =~ s/\303\234/Ue/g;
  $newfile =~ s/\303\244/ae/g;
  $newfile =~ s/\303\266/oe/g;
  $newfile =~ s/\303\274/ue/g;
  $newfile =~ s/\303\237/ss/g;

  #convert to latin1
  if($unidec){
    $newfile = unidecode($newfile);
  }

	#add more translations here:
	
	$newfile =~ s/\\//g;	   #remove backspaces
	$newfile =~ s/\*/x/g;    #windows doesn't like this at all :-)
	$newfile =~ s/&/_and_/g; #ampersand to english
  $newfile =~ s/@/_at_/g;
  $newfile =~ s/['"`]//g;  #remove these completely

	#lowercase some known extensions	
	$newfile =~ s/bat$/bat/gi;
	$newfile =~ s/exe$/exe/gi;
	$newfile =~ s/ogg$/ogg/gi;
	$newfile =~ s/mp3$/mp3/gi;
	$newfile =~ s/rar$/rar/gi;
	$newfile =~ s/pdf$/pdf/gi;
	$newfile =~ s/pdb$/pdb/gi;

	#German Umlauts (Windows charset)
	$newfile =~ s/\x8e/Ae/g;
	$newfile =~ s/\x99/Oe/g;
	$newfile =~ s/\x9A/Ue/g;
	$newfile =~ s/\x84/ae/g;
	$newfile =~ s/\x94/oe/g;
	$newfile =~ s/\x81/ue/g;
	$newfile =~ s/\xe1/ss/g;
	$newfile =~ s/\253/.5/g;

	$newfile =~ s/_\././g;
	$newfile =~ s/\.\././g;
	$newfile =~ s/\357/'/g;

  #Remove all chars we don't want
  if($OPTS{'e'}){
    $newfile =~ s/[^A-Za-z_0-9\.\-]/_/g;
  }else{
    $newfile =~ s/[^A-Za-z_0-9\(\)\[\]\.\-]/_/g;
  }

  #some cleanup
	$newfile =~ s/_-_/-/g;	 #Dashes should not be surounded by underscores
	$newfile =~ s/_-/-/g;
	$newfile =~ s/-_/-/g;
	$newfile =~ s/__+/_/g;    #Reduce multiple spaces to one

  #lowercase if wanted
  $newfile = lc($newfile) if($OPTS{'l'});

	if ("$path/$file" ne "$path/$newfile"){
	  print STDERR "Renaming '$file' to '$newfile'";
	  if (-e "$path/$newfile"){
            print STDERR "\tSKIPPED new file exists\n";
	  }else{
 	    if (rename("$path/$file","$path/$newfile")){
	      print STDERR "\tOKAY\n";
	    }else{
	      print STDERR "\tFAILED\n";
	    }
          }
	}
}

###########################################################
# Read a given directory and its subdirectories
sub readFiles($) {
  (my $path)=@_;
  
  opendir(ROOT, $path);
  my @files = readdir(ROOT);
  closedir(ROOT);

  foreach (@files) {
    next if /^\.|\.\.$/;  #skip upper dirs
    my $file =$_;
    my $fullFilename    = "$path/$file";
    
    
    if (-d $fullFilename) {
      readFiles($fullFilename); #Recursion
    }
    
    renameFile($path,$file); #Rename

  }
}

###########################################################
# prints a short Help text
sub help() {
print <<STOP

      Syntax: sanity.pl [options] <file(s)>

      This tool renames files back to sane names. It does so by replacing
      spaces, german umlauts and some special chars by underscores.

      If a renamed version of a file already exists the renaming will be
      skipped.

      Options:

        -l convert to lowercase
        -e extended cleaning (removes brackets as well)

      The argument can be files and directories. WARNING: Directories will
      be recurseively changed.
      ________________________________________________
      sanity.pl - Sanitize Filenames
      Copyright (C) 2003-2005 Andreas Gohr <andi\@splitbrain.org>

      This program is free software; you can redistribute it and/or
      modify it under the terms of the GNU General Public License as
      published by the Free Software Foundation; either version 2 of
      the License, or (at your option) any later version.
      
      See COPYING for details
STOP
}

##########################################################
# Main

if (@ARGV < 1){
  &help();
  exit -1;
}

foreach my $arg (@ARGV){
  if(-d $arg){
    &readFiles($arg);
  }else{
    &renameFile(dirname($arg),basename($arg));  
  }
}
\end{lstlisting}

