\section{Adaptación de ficheros a DokuWiki}
\label{sanity}

\subsection{Introducción}

En la sección \ref{Samba}. \nameref{Samba} de este documento se explicó como compartir en red algunas carpetas del servidor. Concretamente se establecieron como carpetas compartidas las que guardan las páginas y los archivos subidos al Dokuwiki, \texttt{/data/pages} y \texttt{/data/media} respectivamente.

Sin embargo, para evitar problemas de acceso a las páginas y los archivos creados de forma manual, éstos deben pertenecer al mismo usuario que si hubieran sido creados a través del DokuWiki, es decir, \textit{www-data}.

Adicionalmente, para que los archivos contenidos en la carpeta \texttt{/media} sean enlazables desde el DokuWiki, sus nombres de archivo no deben contener caracteres no soportados, como espacios en blanco o letras mayúsculas.

Para cumplir con estos requerimientos se automatizará la ejecución de un script que cambiará los nombres de archivo por otros más convenientes y cambiará el usuario al que pertenecen, siempre que esto sea necesario.

\subsection{Script para modificar nombres de ficheros}

Para modificar los nombres de los ficheros se utilizará el script \textit{sanity.pl}, escrito en Perl por Andreas Gohr, desarrollador principal de DokuWiki, y distribuido bajo una licencia libre.

El script es recursivo, y modifica los nombres de los archivos situados en la carpeta indicada convirtiendo todos los caracteres a minúsculas, sustituyendo los espacios en blanco por barras bajas, eliminando caracteres no permitidos y evitando que los guiones estén rodeados de rayas bajas. De esta forma DokuWiki no tendrá problemas para leer los ficheros subidos.

Se puede consultar el código del script en la sección \ref{script-sanity}. \nameref{script-sanity}.

\subsection{Script de shell}

Para posteriormente poder automatizar el proceso de ejecución del script de Perl \textit{sanity.pl} y la modificación del dueño de los archivos se creará el script de shell \texttt{sanity.sh}.

\begin{listing}[style=consola, numbers=none]
 # nano /usr/bin/sanity.sh
\end{listing}

\begin{listing}
#!/bin/bash
cd /var/www/dokuwiki/data 
chown -R www-data media
chown -R www-data pages
perl sanity.pl -l media
\end{listing}

Este script considera que \textit{sanity.pl} se encuentra en la carpeta en que se guardan los datos del DokuWiki, es decir \texttt{/var/www/dokuwiki/data}.

Antes de poder ejecutar el script habrá que hacerlo ejecutable.

\begin{listing}[style=consola, numbers=none]
 # chmod +x /usr/bin/sanity.sh
\end{listing}

Una vez hecho esto ya se podrá ejecutar el script escribiendo lo siguiente en la consola:

\begin{listing}[style=consola, numbers=none]
# sanity.sh
\end{listing}

El resultado de la ejecución de este script cuando hay archivos nuevos pendientes de modificar el nombre sería similar a lo siguiente:

\begin{SaveVerbatim}{codigo}
Renaming '5- Manual de usuario.pdf' to '5-manual_de_usuario.pdf'        OKAY
Renaming '1- Memoria.pdf' to '1-memoria.pdf'    OKAY
Renaming '2- Presupuesto.pdf' to '2-presupuesto.pdf'    OKAY
Renaming 'Jorge Cuervo Ortega' to 'jorge_cuervo_ortega' OKAY
Renaming 'FA_213_dDa.txt' to 'fa_213_dda.txt'   OKAY
Renaming 'FV_171_dDa.txt' to 'fv_171_dda.txt'   OKAY
Renaming 'FV_236_dDa.txt' to 'fv_236_dda.txt'   OKAY
Renaming 'FV_233_dDa.txt' to 'fv_233_dda.txt'   OKAY
Renaming 'filtrado_ECG.m' to 'filtrado_ecg.m'   OKAY
Renaming 'FV_241_2_dDa.txt' to 'fv_241_2_dda.txt'       OKAY
Renaming 'FV_246_dDa.txt' to 'fv_246_dda.txt'   OKAY
Renaming 'FV_234_dDa.txt' to 'fv_234_dda.txt'   OKAY
\end{SaveVerbatim}
\respuesta{codigo}

\subsection{Automatización del proceso}

Para que el script descrito en la sección anterior se ejecute de forma automática habrá que crear un archivo en \texttt{/etc/cron.d}, llamado por ejemplo \texttt{sanity}.

A continuación se muestra un ejemplo del contenido de este fichero para que el script se ejecute cada dos minutos de ocho de la mañana a diez de la noche y de lunes a viernes.

\begin{listing}[style=consola, numbers=none]
# nano /etc/cron.d/sanity
\end{listing}

\begin{listing}
#min hora dia mes dia_semana usuario comando
*/2 08-22 * * 1-5 root /usr/bin/sanity.sh
\end{listing}

\newpage
\subsection{Resumen}

\begin{enumerate}
 \item  Creación del script \texttt{sanity.pl} (ver código en \ref{script-sanity}. \nameref{script-sanity})
\begin{listing}[style=consola, numbers=none]
# nano /var/www/dokuwiki/data/sanity.pl
\end{listing}

\begin{listing}[style=consola, numbers=none]
 # chmod a+x /usr/bin/sanity.sh
\end{listing}

\item Automatización de la ejecución del script

\begin{listing}[style=consola, numbers=none]
 # nano /etc/cron.d/sanity
\end{listing}

\begin{listing}[style=archivo, numbers=none]
 #min hora dia mes dia_semana usuario comando
*/10 08-22 * * 1-5 root /usr/bin/sanity.sh
\end{listing}

\end{enumerate}