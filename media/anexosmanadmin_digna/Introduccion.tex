\section{Introducción}
\label{introduccion}

Este documento contiene una serie de manuales que describen la instalación y configuración del sistema, y tiene dos funciones principales: Servir como guía para la implementación de otros sistemas similares (como la \emph{Red Temática de la Comunidad de Trabajo de los Pirineos}) y describir la arquitectura software del sistema y el modo de configurar sus distintos bloques para que el administrador lo pueda usar como referencia a la hora de hacer modificaciones y ajustes en la intranet.

Cada sección de este documento describirá la instalación y configuración de un bloque distinto de la arquitectura software del sistema. Algunas de las secciones dispondrán de dos partes; en la primera de ellas se hará una descripción detallada del proceso a seguir y al final de la misma se incluirá un apartado \emph{resumen} que recogerá de forma esquemática las acciones a llevar a cabo.

Algunas de las indicaciones mostradas en este manual tienen un formato especial. Por ejemplo, para indicar comandos a introducir en un terminal de consola se utilizará fondo oscuro. Los comandos estarán precedidos por el símbolo \$ o por \# en caso de que haya que ejecutarlos como root.

\begin{listing}[style=consola, numbers=none]
$ comando a ejecutar en consola
# comando a ejecutar como root
\end{listing}

Cuando se quiera mostrar la respuesta devuelta al ejecutar un comando en la consola, se hará de la siguiente forma:

\begin{SaveVerbatim}{codigo}
Esto es la respuesta a los comandos que hemos ejecutado
\end{SaveVerbatim}
\respuesta{codigo}

También se usará un formato especial para mostrar el contenido de un fichero o script.
\begin{listing}
Este es el contenido de un fichero.
Se puede mostrar el fichero indicadores del numero de linea o sin ellos.
\end{listing}

Cuando se considere que no es necesario haber leído un script para entender el proceso de instalación o configuración del bloque al que pertenece, éste se incluirá en la sección \ref{scripts}. \nameref{scripts}. Si por el contrario es un script importante que conviene tener delante al leer el manual o es muy corto y se considera que no entorpece el proceso de lectura, se insertará directamente en la explicación.