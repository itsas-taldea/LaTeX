\usepackage[utf8]{inputenc} %Establece codificación a utf8 (para poder escribir tildes)
\usepackage[spanish]{babel} %Traducir títulos a castellano
%Paquetes para f\'ormulas
% \usepackage{amsmath}
% \usepackage{amsfonts}
% \usepackage{amssymb}
%enlaces
\usepackage{url} %Para generar enlaces
\usepackage[breaklinks=true, colorlinks]{hyperref} %Para generar enlaces
%colores
\usepackage{xcolor}
\usepackage{graphicx} %Para trabajar con gráficos
%listados de codigo
\usepackage{listings} %Para listar código fuente
\lstloadlanguages{[LaTeX]TeX}
%Tablas
\usepackage{threeparttable} %Para poner pies de tabla
\usepackage{booktabs} %Para usar toprule, midrule, etc en tablas
\usepackage{subfig} %Para insertar subtablas o subfiguras
\usepackage{multirow} %Para poder usar multicolumn y multirow
\usepackage{colortbl} %Para usar colores en las tablas
% \usepackage{color} %dependencia de colortbl
% \usepackage{array} %dependencia de colortbl
%Otros paquetes
\usepackage{mdwlist} %Define el entorno  basedscript, similar a description
\usepackage{mhchem} %Para escribir fórmulas químicas 
%\chapterstyle{madsen} %Define el estilo de los capítulos de memoir
\author{Digna González Otero}
\title{Información adicional \LaTeX}

\setlength{\parindent}{0cm} %Cambia la indentaci\'on de primera l\'inea
\setlength{\parskip}{8pt} %Cambia la separaci\'on entre p\'arrafos

% Establecemos los valores por defecto de Listings
\lstset{
	language={[LaTeX]TeX},						% Lenguaje por defecto
	%
	% estilos
	keywordstyle=\bfseries\ttfamily\color[rgb]{.8,.1,.2},	% estilos de palabras clave, identificadores, etc...
	identifierstyle=\ttfamily,
	commentstyle=\color[rgb]{0.1,0.5,0.1},			 
	stringstyle=\ttfamily\color[rgb]{0.2,0.2,.7},			
	basicstyle=\footnotesize, 						% the size of the fonts used for the code 
	%
	% numeración (desabilitadas en este momento :-)
	%numbers=left, 								% where to put the line-numbers 
	numberstyle=\footnotesize, 					% size of the fonts used for the line-numbers 
	stepnumber=1, 							% the step between two line-numbers. 
	numbersep=5pt, 							% how far the line-numbers are from the code 
	%
	% espacios
	showspaces=false, 							% show spaces adding particular underscores 
	showstringspaces=false,	 					% underline spaces within strings 
	showtabs=false, 							% show tabs within strings through particular underscores 
	tabsize=6,									% sets default tab-size to 2 spaces
	%
	% cuadro
	backgroundcolor=\color{white}, 				% sets background color (needs package) 
	frame=single, 								% adds a frame around the code
	rulecolor=\color[rgb]{.3,.3,.3},					% set the frame's color. 
	captionpos=b, 								% sets the caption-position to bottom 
	%
	% line breaking
	breaklines=true, 							% sets automatic line breaking 
	breakatwhitespace=false, 					% automatic breaks happen at whitespace 
	prebreak = \raisebox{0ex}[0ex][0ex]{\ensuremath{\hookleftarrow}}, % Nos dibuja una flecha ``guay'' cuando el código no entra en una linea
	escapeinside=@@,		% Para escapar a LaTeX. los acentos
}

\renewcommand*{\contentsname}{Tabla de contenidos}
\renewcommand{\tablename}{Tabla} %Para que se en el caption de las tablas ponga
% Tabla y no Cuadro
