\documentclass[a4paper,titlepage,10pt,oneside]{article}

\usepackage[utf8]{inputenc} % Para poder escribir acentos directamente
\usepackage[spanish]{babel} % Para que LaTeX sepa que el texto está en español y divida las sílabas al final de una línea correctamente, entre otras cosas
\usepackage[left=3cm, right=3cm, top=3cm, bottom=3cm]{geometry} % Para ajustar los márgenes, ya que por defecto son mayores
\usepackage{graphicx} % Para insertar imágenes
\usepackage{fancyhdr} % Para definir encabezados y pies de página personalizados

\pagestyle{fancy}

\fancyhead[L]{\textbf{Proyecto de Domótica} \\ \small{LOGO! de Siemens}} % Páginas pares e impares, alineado a la izquierda en la cabecera
\fancyhead[R]{\textbf{Unai Martinez Corral} \\ \small{unai.martinez@udc.es}} % Páginas pares e impares, alineado a la derecha en la cabecera
\fancyfoot[C]{\textbf{\thepage}} % Páginas pares e impares, centrado en el pie
\fancyfoot[L]{EUP Ferrol} % Páginas pares e impares, alineado a la izquierda en el pie
\fancyfoot[R]{2009-2010} % Páginas pares e impares, alineado a la derecha en el pie

\renewcommand{\footrulewidth}{0.4pt} % Grosor de la línea de separación del footer

\usepackage[usenames,dvipsnames]{color} % Para disponer de 59 colores estandar

\usepackage[
    bookmarks=true, % Mostrar la herramienta de marcadores al mostrar el documento
    unicode=true, % Acepta caracteres no latinos
    pdftitle={Proyecto de Domotica}, % Metadatos fichero resultante (Título)
    pdfsubject={Sistema por controlador programable LOGO! de Siemens}, % Metadatos fichero resultante (Asunto)
    pdfauthor={Unai Martinez Corral}, % Metadatos fichero resultante (Autor)
    pagebackref=true,
    colorlinks=true, % Enlaces de colores
    linkcolor=BlueViolet, % Color de enlace interno
    urlcolor=RoyalBlue, % Color de enlace externo
    ]{hyperref}

%--------------------------------------------------------------------------

\title{ % Definimos el título
\Huge{\textbf{Proyecto de Dom\'otica}} \\
\vspace{0.5cm}
\Large{Sistema por controlador programable LOGO! de Siemens}
}

\author{ %Definimos autor
\Large{Unai Martinez Corral} \\
\small{unai.martinez@udc.es}}

\date{2009-2010} %Definimos fecha

%--------------------------------------------------------------------------

\begin{document}

\maketitle % Mostramos el título, autor y fecha

\tableofcontents % Mostramos el índice

%--------------------------------------------------------------------------

% Incluimos los ficheros donde se encuentran las diversas secciones

\newpage
\section{Introducci\'on}

Este documento constituye la memoria del proyecto desarrollado para la asignatura \href{https://campusvirtual.udc.es/guiadocente/guia_docent/index.php?centre=770&ensenyament=770611&assignatura=770611541&idioma=}{Dom\'otica} impartida en la \href{http://lucas.cdf.udc.es/}{Escuela Universitaria Poli\'ectina de Ferrol}, centro perteneciente a la \href{http://www.udc.es}{Universidade da Coruña}. Dicho proyecto consiste en el desarrollo de un sistema de control para el riego exterior y el alumbrado de una vivienda unifamiliar, de acuerdo con el pliego de condiciones establecido por el profesor titular, Antonio Masdias y Bonome.

El sistema deber\'a estar pensado para su implementaci\'on en un \hyperref[LOGO]{M\'odulo L\'ogico LOGO!} de \hyperref[Siemens]{Siemens}, y se desarrollar\'a con el software \href{http://www.automation.siemens.com/mcms/programmable-logic-controller/en/logic-module-logo/logo-software/Pages/Default.aspx}{LOGO! Soft Comfort} del mismo fabricante.

\subsection{Pliego de condiciones}

La vivienda sobre la que se desarrollar\'a tiene tres plantas: semis\'otano, planta baja y primera planta. Dispone de un terreno ajardinado alrededor de la casa y un h\'orreo en dicho terreno, al cual se deber\'a aplicar una iluminaci\'on nocturna.
El jard\'in se deber\'a dividir en cuatro sectores de riego:

\begin{center}
\begin{tabular}[c]{c|c}
  Denominaci\'on & Tiempo de regado \\
  \hline
  Flores 1 & 5 minutos \\
  Césped 1 & 7 minutos \\
  Césped 2 & 9 minutos \\
  Césped 3 & 11 minutos \\
\end{tabular}
\end{center}

De forma autom\'atica, se deber\'a encender y apagar el alumbrado del h\'orreo, as\'i como regar, todas las noches a las 23:30 horas. Independientemente, la iluminaci\'on exterior se podr\'a accionar manualmente desde dentro de la vivienda en todo momento.

\subsection{Caracter\'isticas adicionales}

Como caracter\'isticas adicionales definidas por el desarrollador, encontramos las siguientes:

\begin{itemize}
 \item El usuario necesitar\'a \'unicamente dos pulsadores, o uno doble, para interactuar con el sistema. Uno de ellos afectar\'a al riego y el otro a la iluminaci\'on.
 \item El usuario tendr\'a la posibilidad de interrumpir el ciclo de regado en caso de que quisiera hacer uso del terreno ajardinado de 23:30 a 23:41.
 \item Si lo considera necesario, el usuario podr\'a iniciar un ciclo de regado en cualquier momento, si bien los tiempos de duraci\'on ser\'an los especificados en el pliego de condiciones.
 \item En lugar de iniciar un ciclo completo de regado, el usuario podr\'a activar el sistema de riego y desactivarlo cuando lo considere adecuado.
 \item Si se ha iniciado un ciclo de regado manualmente, o se ha activado el sistema de riego, antes de las 23:30 y \'este sigue en funcionamiento, se anular\'a el ciclo programado.
 \item El usuario podra desconectar manualmente la alimentaci\'on tanto de las v\'alvulas de riego como de la iluminaci\'on en caso de emergencia. Esta funcionalidad no se ha implementado por medio del controlador programable, sino de forma el\'ectrica.
 \item La iluminaci\'on del h\'orreo permanecer\'a activa hasta las 7:30, salvo que el usuario la desactive manualmente.
\end{itemize}

\newpage
\section{An\'alisis l\'ogico y funcionamiento del sistema}

A la hora de realizar el an\'alisis l\'ogico de la programaci\'on desarrollada para el LOGO!, tomaremos como base el diagrama y la documentaci\'on generados por el software utilizado. Estos documentos se encuentran anexos bajo el nombre '\hyperref[CircuitDiagram]{Circuit Diagram}'. Asimismo, describiremos por separado el funcionamiento autom\'atico, la inicializaci\'on manual del sistema de riego y el accionamiento manual de la iluminaci\'on.

\subsection{Funcionamiento autom\'atico}

En condiciones normales, cuando el usuario no interaccione con el sistema, los bloques que afectan al funcionamiento de \'este son:

\begin{itemize}
 \item Salidas (Output): Q1, Q2, Q3, Q4 y Q5.
 \item Retardo a la desconexi\'on (Off-delay): B006, B007, B008, B012 y B015.
 \item Temporizador semanal (Weekly timer): B005.
 \item Puerta l\'ogica AND: B010.
 \item Puerta l\'ogica NOT: B017.
\end{itemize}

El temporizador semanal se encuentra programado para que genere un pulso \footnote{\tiny{La opci\'on que permite al temporizador semanal generar un pulso como salida s\'olo est\'a disponible en el modelo 0BA6. Se ha optado por activar la salida del temporizador a las 23:30 y desactivarla a las 23:31 para poder utilizar modelos inferiores. Pese a ello, el circuito formado por la puerta AND y la NOT provoca que s\'olo se genere un pulso.}} todos los d\'ias a las 23:30. Este pulso, conectado a la entrada 'Trigger' de todos los bloques de retardo a la desconexi\'on, activa directamente las salidas, dado que cada uno de los bloques de retardo tiene una asociada. El tiempo que pemancer\'an activas las señales tras haber recibido el pulso generado por el temporizador depender\'a de los par\'ametros dados en el pliego de condiciones. La siguiente tabla muestra la salida asociada a cada bloque de retardo, la duraci\'on programada y el sector de riego al que afecta o si controla la iluminaci\'on:

\begin{center}
\begin{tabular}{c|c|c|c}
 Bloque de retardo&Salida&Tiempo&Sector\\
 \hline
 B015&Q1&5min&Flores 1 \\
 B006&Q2&7min&Césped 1 \\
 B007&Q3&9min&Césped 2 \\
 B008&Q4&11min&Césped 3 \\
 B012&Q5&8h&Iluminaci\'on \\
\end{tabular}
\end{center}

La puerta NOT invierte la señal de la salida Q4, el riego de mayor duraci\'on, y se encuentra conectada a una de las entradas de la puerta AND. La salida del temporizador semanal ocupa otra de las entradas y es la salida de la puerta la que se conecta a los bloques de retardo de la desconexi\'on. Este pequeño circuito auxiliar bloquea el pulso generado por el temporizador en caso de que se haya inicializado el sistema de riego manualmente y \'este se encuentre en funcionamiento.

\subsection{Accionamiento manual de la iluminaci\'on}

Uno de los pulsadores se encuenta cableado directamente al bloque de entrada (Input) identificado como I2. Para poder simular correctamente el funcionamiento del sistema, se ha parametrizado este elemento, activando la opci\'on 'Momentary pushbutton (make)' en la pestaña 'Simulation'. La entrada se encuentra conectada al conector 'Trigger' de un relay de pulsos (Pulse Relay) cuyo identificador es B020. La salida de este bloque afecta directamente al bloque de salida Q5, que controla la iluminaci\'on.

Para hacer compatibles el accionamiento manual y el funcionamiento autom\'atico, las salidas de los bloques B012 (Off-delay) y B020 (Pulse Relay) se han conectado a la entrada de una puerta OR identificada como B011, es la salida de \'esta la que se encuentra realmente conectada a Q5. Adem\'as, la señal del pulsador est\'a conectada al bloque de retardo de la desconexi\'on de la iluminaci\'on, lo cual provoca que, ante la intervenci\'on del usuario de 23:30 a 7:30, se anule el retardo y la iluminaci\'on s\'olo sea gestionada por el pulsador.

El montaje descrito impide que el usuario pueda inicializar la activaci\'on autom\'atica de la iluminaci\'on una vez ha interactuado con el sistema a partir de las 23:30. El autor considera que, en principio, la iluminaci\'on no interfiere en el uso del terreno ajardinado y el usuario no deber\'ia tener inter\'es en desconectarla en un momento dado para despu\'es continuar con la programaci\'on. En caso de querer hacerlo, tiene la posibilidad de cortar la alimentaci\'on mediante los componentes el\'ectricos incluidos e independientes del sistema programable. Como se expondr\'a en el siguiente apartado, en el caso del control de riego se ha implementado la posibilidad de interrumpir la señal del bloque que retarda la desconexi\'on, e incluso de iniciarla en otro momento. Considera tambi\'en que resulta redundante la implementaci\'on de una soluci\'on id\'entica. La adici\'on de otras soluciones, como puede ser añadir un segundo temporizador semanal que mantenga su señal activa de 23:30 a 7:30 y que \'esta pueda ser habilitada o deshabilitada mediante pulsaciones largas del usuario, no constituyen un reto en el aprendizaje del uso de las herramientas, pues los recursos a utilizar ya est\'an incluidos en el proyecto. Valora que hacerlo complicar\'ia innecesariamente el an\'alisis y comprensi\'on del sistema, por lo que ha optado por no hacerlo.

\subsection{Accionamiento manual del sistema de riego}

En el caso del accionamiento manual del sistema de riego, la programaci\'on contempla la posibilidad de que el usuario efect\'ue una pulsaci\'on corta o una prolongada. La distinci\'on entre ambas se lleva a cabo mediante los siguientes bloques:

\begin{itemize}
 \item Entrada (Input): I1.
 \item Relay de pulsos (Pulse Relay): B016.
 \item Retardo a la conexi\'on memorizado (Retentive On-Delay): B014.
 \item Flag: M1.
 \item Retardo a la conexi\'on (On-Delay): B013 y B018.
 \item Retardo a la desconexi\'on (Off-delay): B009.
 \item Puerta l\'ogica OR: B021.
\end{itemize}

Si la duraci\'on de la pulsaci\'on es menor que 0:30 segundos, el bloque B018 impide que la señal llegue a los bloques de retardo B009 y B013. En estas condiciones, pasados 0:30 segundos, se activa la salida del bloque de retardo a la conexi\'on memorizado B014. La activaci\'on de \'esta provoca un cambio en el estado de la salida del relay de pulsos B016. Al mismo tiempo, el flag M1 provoca que se desactive la salida del bloque B014.

La salida del relay de pulsos afecta directamente a las salidas Q1, Q2, Q3 y Q4, adem\'as de a las señales de reset de los bloques de retardo a la desconexi\'on analizados en el apartado 'Funcionamiento autom\'atico' de esta misma secci\'on. Si nos encontramos entre las 23:41 y las 23:30 del d\'ia siguiente, el sistema de riego se pondr\'a en marcha. Si estuvi\'eramos entre las 23:30 y las 23:41, adem\'as de continuar en funcionamiento, se desactivar\'a la temporizaci\'on autom\'atica. En cualquier caso, ser\'a necesario repetir la operaci\'on para desactivarlo, es decir, volver a realizar una pulsaci\'on corta en la entrada I1.

Si la duraci\'on de la pulsaci\'on es mayor que 0:30 segundos, se activa la salida del bloque de retardo a la conexi\'on B018. \'Esta activa el bloque de retardo a la desconexi\'on B009, reseteando la salida del relay de pulsos B016, con la consiguiente desconexi\'on del sistema de riego, en caso de que estuviera activado. La señal que resetea el relay permanecer\'a activa durante 0:35 segundos \footnote{\tiny{La resoluci\'on permitida en modelos inferiores al 0BA4 es de 0:05 segundos.}} despu\'es de soltar el pulsador, por lo que, independientemente de la duraci\'on del pulso a partir de 0:30 segundos, el circuito compuesto por los bloques B014 y M1 no afecta al sistema.

En caso de que la pulsaci\'on se prolongue hasta alcanzar los 3:00 segundos, se activa la salida del bloque de retardo a la conexi\'on B013. \'Esta afecta directamente a los bloques de retardo a la desconexi\'on de las salidas de riego, activando su temporizaci\'on.

\subsection{Manual de usuario}

Autom\'aticamente se activar\'an el sistema de riego y la iluminaci\'in del h\'orreo todos los d\'ias a las 23:30. Cada uno de los elementos permanecer\'a encendido durante el tiempo programado. Si quiere interrumpir el ciclo de riego o desconectar la iluminaci\'on, efect\'ue dos pulsaciones cortas en el pulsador correspondiente (recuerde que dispone de dos, uno de ellos para la iluminaci\'on y otro para el riego).

Si quiere encender el sistema de riego o la iluminaci\'on indefinidamente, efect\'ue una pulsaci\'on corta en el pulsador. Una nueva pulsaci\'on los desactivar\'a. Tenga en cuenta que, si el sistema de riego est\'a activado a las 23:30, la temporizaci\'on diaria no se activar\'a.

Si quiere iniciar un ciclo de riego en cualquier otro momento del d\'ia, mantenga el pulsador de riego accionado durante, al menos, 3 segundos. Para detenerlo, al igual que cuando se ha iniciado autom\'aticamente, realice dos pulsaciones cortas.

\newpage
\section{Componentes}

\subsection{M\'odulo l\'ogico LOGO!}

Debido al tipo de cargas a controlar, ser\'a requisito indispensable que el modelo escogido tenga salidas por rel\'e, y no por transistor, para evitar tener que introducir elementos externos. Asimismo, el n\'umero de \'estas que necesitamos (cinco) ser\'a otra caracter\'istica a tener en cuenta. En lo que a la tensi\'on de alimentaci\'on se refiere y la naturaleza de \'esta, optaremos por un m\'odulo que pueda alimentarse directamente de la red de la casa, lo cual supone que acepte 230 voltios de tensi\'on alterna. Por \'ultimo, dado que el usuario interactuar\'a con el sistema mediante pulsadores, y no directamente sobre el LOGO!, no es necesario que \'este incorpore pantalla, ya que la programaci\'on se llevar\'a a cabo mediante un ordenador.

El modelo que cumple todas estas caracter\'isticas es el \hyperref[LOGO]{LOGO! 230RCO} (8 ED\footnote{\tiny{Entradas digitales}}, 4 SD\footnote{\tiny{Salidas digitales}}, 85-253V AC) cuya referencia es 6ED1052-2FB00-0BA1, y el m\'odulo de expansi\'on DM8 230R con referencia 6ED1055-1FB00-0BA1 que ofrece 4 ED y 4 SD m\'as, tambi\'en por rel\'e.

\subsection{Pulsadores}

En principio cualquier pulsador comercial disponible para instalaciones el\'ectricas dom\'esticas comunes es v\'alido para el sistema planteado. Dado que \hyperref[Siemens]{Siemens} tiene una amplia gama de este tipo de dispositivos, se ha optado por el modelo \hyperref[Pulsadores]{5TD2 111} de su catalogo ET D1 2010. Para este mismo soporte encontramos embellecedores con referencia 5TG7 tanto de la gama 'DELTA profil' como 'DELTA style' y 'DELTA natur', disponibles en diferentes colores y con variadas serigraf\'ias. En lo que respecta al marco, tenemos a nuestra disposici\'on diversos modelos de las mismas gamas bajo la referencia com\'un 5TG1.

Debido a la baja impedancia de este tipo de dispositivos, aunque s\'olo se ha contemplado la instalaci\'on de un pulsador doble a fin de facilitar el desarrollo del proyecto, se podr\'ian utilizar dos pulsadores independientes, e incluso contar con m\'as de un pulsador que cumplan la misma funci\'on, en caso de que, por comodidad, el usuario final quisiera poder accionar los mecanismos desde diferentes puntos de la vivienda. Su instalaci\'on no revestir\'ia ninguna complicaci\'on, pues ir\'ian conectados en paralelo a la entrada correspondiente del LOGO!. El accionamiento de cualquiera de \'estos enviar\'ia la orden correspondiente el m\'odulo l\'ogico, sin necesidad de ser reprogramado para ello.

\subsection{Iluminaci\'on h\'orreo}

Como sucediera en el caso de los pulsadores, la gama de productos de diferentes fabricantes disponible para iluminaci\'on de exteriores es muy amplia y variada. Para esta aplicaci\'on concreta, que requiere una iluminaci\'on focalizada, pero no muy intensa, se han considerado soluciones basadas en LEDs. En el caso del fabricante \hyperref[OSRAM]{OSRAM}, encontramos el modelo '\hyperref[LED]{HYDROSTAR MINI OSTAR}' de dimensiones reducidas y muy bajo consumo (12W). Para iluminar adecuadamente el h\'orreo, utilizaremos cuatro de estos productos. Asimismo, para alimentarnos necesitaremos dos conversores \hyperref[OT]{OT 35/200-240/700} (35W).

Como añadido, se iluminar\'a con pequeños puntos de luz el camino desde el porche hasta \'este. El modelo a utilizar ser\'a el '\hyperref[LED]{NAUTILUS PICO 1 LED}' (0.8-1.2W), tambi\'en de OSRAM, con un conversor \hyperref[OT]{OT 9/200-240/350} (9W). Dado que estos puntos se conectar\'an en paralelo con la iluminaci\'on del propio h\'orreo, no ser\'a necesario realizar ninguna modificaci\'on en la programaci\'on del m\'odulo l\'ogico. El consumo de todos los componentes dista mucho de alcanzar el m\'aximo soportado por las salidas del LOGO!, por lo que no habr\'a riesgo de fallo.

\subsection{Aspersores y difusores}

Debido a la diferencia de tamaños entre los sectores de riego y a la ubicacion de los componentes de riego -podemos verlos en el anexo '\hyperref[Planos]{Planos}'-, deberemos utilizar tanto aspersores como difusores para satisfacer las necesidades del diseño. Para este sistema en concreto, se ha optado por dos productos del fabricante \hyperref[RainBird]{RainBird}, como son los difusores de la serie \hyperref[1400]{1400} para aquellos sectores con un radio de alcance menor a 0,9m, los difusores de la serie \hyperref[1800]{1800} para sectores con un radio de hasta 5,5m y los aspersores de la serie \hyperref[3500]{3500} para los que tengan un radio mayor. El proyecto contempla el uso de dos aspersores para el sector de riego denominado 'C\'esped 1', cuatro difusores de la serie 1800 para los sectores 'C\'esped 2' y 'C\'esped 3', y tres difusores de la serie 1400 para el sector 'Flores 1'.

Como puede verse en los planos, la ubicaci\'on de \'estos se ha definido con el objetivo de interferir lo m\'inimo posible en el uso del espacio, situ\'andolos en aquellos puntos de menor tr\'ansito. El hecho de que todos ellos permitan la selecci\'on del \'angulo de riego y tengan un alcance tambi\'en configurable, facilita la citada disposici\'on.

\subsection{Electrov\'alvulas}

Tomando en cuenta los difusores y aspersores escogidos en el apartado anterior, \'estos son los caudales y la presi\'on que necesitaremos para cada uno de los sectores:

\begin{center}
\begin{tabular}{c|c|c}
 Sector&Caudal (min.)[$l \over h$]&Presi\'on (min)[bares]\\
 \hline
 Flores 1&60&1.4 \\
 Césped 1&120&1.7 \\
 Césped 2&20&1.0 \\
 Césped 3&20&1.0 \\
\end{tabular}
\end{center}

El mismo fabricante, RainBird, comercializa las el\'ectrov\'alvulas modelo \hyperref[LFV]{LFV-075} que cumplen con creces los requisitos para este sistema: caudal (45,6-1136 $l \over h$) y presi\'on (1,0-10,3 bares). Se alimentan con una tensi\'on alterna de 24V y su consumo de corriente en r\'egimen permanente es de 0.19A, mientras el pico de arranque es de 0.30A.

Teniendo en cuenta el tipo de alimentaci\'on que requieren las electrov\'alvulas y el consumo m\'aximo de \'estas (debemos contar con la peor situaci\'on, cuando activemos todas al mismo tiempo), necesitaremos un tranformador de tensi\'on alterna de 230V a 24V con una potencia m\'inima de 28.8VA. En el catalogo de \hyperref[Polylux]{Polylux} encontramos el modelo con referencia PB40 que cumple las especificaciones, ofreci\'endonos una potencia m\'axima de 40VA. Trabaja con frecuencias de 50 a 60 Hz y, adem\'as, se ofrece en un encapsulado adecuado para su montaje en carril DIN, lo cual facilita su integraci\'on.

\subsection{Magnetot\'ermicos}

Como medida activa de seguridad, y con el fin de poder cortar la alimentaci\'on tanto de la iluminaci\'on como de las electrov\'alvulas, en caso de que el usuario as\'i lo quiera, se incorporar\'an sendos magnetot\'ermicos en el sistema. Siemens, una vez m\'as, pone a nuestra disposici\'on en su cat\'alogo el modelo con referencia \hyperref[Magnetotermicos]{5SJ3106-7}, que consiste en un interruptor de un \'unico polo que soporta 6A de corriente y permite su montaje en carril DIN. Incorporaremos un dispositivo de este tipo entre la alimentaci\'ion y los contactos de los componentes de iluminaci\'on. Otro, se situar\'a entre la alimentaci\'on y el transformador al que conectaremos posteriormente las electrov\'alvulas.

\newpage
\section{Atribuciones y licencia}

El diagrama y la informaci\'on contenida en el anexo '\hyperref[CircuitDiagram]{Circuit Diagram}', han sido generados autom\'aticamente a partir del proyecto realizado por el autor a trav\'es del programa LOGO! Soft Comfort desarrollado por Siemens. \\

Las vistas en planta y los planos de la vivienda sobre los que se ha dibujado la distribuci\'on de los componentes en el anexo '\hyperref[Planos]{Planos}' han sido facilitados por Antonio Masdias y Bonome. \\

Siemens, OSRAM, Rainbird y Polylux son marcas registradas, y la propiedad tanto de \'estas como la sus productos pertenece a las respectivas empresas:

\begin{itemize}
 \item Siemens Aktiengesellschaft y/o sus afiliados
 \item OSRAM GmbH y/o sus subsidiarios
 \item Rain Bird Corporation, Inc.
 \item Polylux
\end{itemize}

El resto del contenido de este documento, se distribuye bajo licencia Creative Commons By 3.0 (CC-by-3.0). Están permitidas la copia, distribución, y comunicación pública de la obra, así como su modificación y adaptación, siempre y cuando se reconozca la autoría mencionando a Unai Martínez Corral (pero no de una manera que sugiera que tiene su apoyo o apoya el uso que hace de su obra).

\begin{center}
 \includegraphics[width=200pt]{./ccby.png}
\end{center}

El \href{http://creativecommons.org/licenses/by/3.0/legalcode}{texto legal} completo está disponible en la página de la organización \href{http://creativecommons.org}{Creative Commons}:
\begin{center}http://creativecommons.org/licenses/by/3.0/legalcode\end{center}
\section{Bibliograf\'ia}

\begin{itemize}
 \item P\'agina web de \href{http://www.siemens.com}{Siemens}\label{Siemens}
 \begin{itemize}
  \item \href{https://eb.automation.siemens.com/goos/catalog/Pages/ProductData.aspx?catalogRegion=WW&language=en&regionUrl=\%2f&activetab=order&nodeID=10045207#activetab=product&}{Automation technology}. Secci\'on \href{https://eb.automation.siemens.com/goos/catalog/Pages/ProductData.aspx?catalogRegion=WW&language=en&regionUrl=\%2f&activetab=order&nodeID=10000027#activetab=order&}{LOGO!}\label{LOGO}
  \item \href{http://www.automation.siemens.com/MCMS/ELECTRICAL-INSTALLATION-TECHNOLOGY/EN/Pages/Electrical_Installation_Technology.aspx}{Electrical Installation Technology}. Secciones
  \begin{itemize}
   \item \href{http://www.automation.siemens.com/mcms/electrical-installation-technology/EN/products/delta/Functions/Pages/delta_functions_electrical_installation.aspx}{DELTA Switches and Socket Outlets}\label{Pulsadores}
   \item \href{http://www.automation.siemens.com/mcms/electrical-installation-technology/EN/products/beta/Pages/beta_low_voltage_circuit_protection_electrical_installation.aspx}{BETA Low-Voltage Circuit Protection}\label{Magnetotermicos}
  \end{itemize}
 \end{itemize}
 \item P\'agina web de \href{http://www.osram.com}{OSRAM}\label{OSRAM}
 \begin{itemize}
  \item Proffessional Products $>$ Luminaries $>$ Outdoor $>$ \href{http://www.osram.com/osram_com/Professionals/Luminaires/Outdoor/LED_Fixtures/index.html}{LED Fixtures}\label{LED}
  \item Proffessional Products $>$ Electronic Control Gear $>$ \href{http://www.osram.com/osram_com/Professionals/ECG_\%26_LMS/ECGs_for_LED_modules/OT_static/index.html}{OT for static lightint}\label{OT}
 \end{itemize}
 \item P\'agina web de \href{http://www.rainbird.es}{RainBird}\label{RainBird}
 \begin{itemize}
  \item Profesionales del riego $>$ Productos $>$ Riego residencial $>$ Difusores $>$ \href{http://www.rainbird.es/19-6325-Fiche-produit.php?id_produits=24}{Serie 1400}\label{1400}
  \item Profesionales del riego $>$ Productos $>$ Riego residencial $>$ Difusores $>$ \href{http://www.rainbird.es/19-6325-Fiche-produit.php?id_produits=41}{Serie 1800}\label{1800}
  \item Profesionales del riego $>$ Productos $>$ Riego residencial $>$ Aspersores $>$ \href{http://www.rainbird.es/19-6325-Fiche-produit.php?id_produits=3}{Serie 3500}\label{3500}
  \item Profesionales del riego $>$ Productos $>$ Riego residencial $>$ V\'alvulas $>$ \href{http://www.rainbird.es/19-6325-Fiche-produit.php?id_produits=156}{LFV-075}\label{LFV}
 \end{itemize}
 \item P\'agina web de \href{http://www.polylux.com}{Polylux}\label{Polylux}
 \begin{itemize}
  \item \href{http://www.polylux.com/productos.php?categoria=1}{Transformadores}
 \end{itemize}
\end{itemize}

%--------------------------------------------------------------------------

\newpage
\fancyfoot[C]{} % Redefinimos páginas pares e impares, centrado en el pie para dejarlo vacío y que no se muestre el número de página

\begin{center}
 \textcolor{White}{.} \\
 \vspace{9cm}
 \section*{\Huge{\textbf{\underline{Circuit Diagram}}}}\label{CircuitDiagram} %Creamos la sección con asterisco para que no se indexe
\end{center}

%--------------------------------------------------------------------------

\newpage
\fancyfoot[C]{} % Redefinimos páginas pares e impares, centrado en el pie para dejarlo vacío y que no se muestre el número de página

\begin{center}
 \textcolor{White}{.} \\
 \vspace{9cm}
 \section*{\Huge{\textbf{\underline{Planos}}}}\label{Planos} %Creamos la sección con asterisco para que no se indexe
\end{center}

\end{document}