\usepackage[spanish]{babel}   % (rec)  da soporte para castellano a LaTeX
\usepackage[utf8]{inputenc}    % (rec) soporte para tildes
\usepackage{url} %Para poder poner enlaces web
\RequirePackage{ifpdf}%para poder usar if pdf

%Para que haya enlaces internos en el documento
\ifpdf
    \usepackage[%
        pdftex,%
        colorlinks=true,% If false no colorlinks
        pdfborder={0 0 0},% No border around Links
        hyperindex,%
        plainpages=false,%
        bookmarksopen,%
        bookmarksnumbered,%
	breaklinks=true%
        ]{hyperref}
\else
    \usepackage[colorlinks=false,pdfborder={0 0 0}]{hyperref}
\fi

\usepackage{xcolor} %Para poder usar colores

%Cargo el paquete glossaries con diferentes opciones: 
% * acronym para crear listas de acr\'onimos
% * nonumberlist para que en la lista de acr\'onimos no ponga en qu\'e p\'aginas
% aparece  cada t\'ermino
% * shortcuts para poder utilizar los nombres cortos para acr\'onimos: \ac, etc
% * xindy para que utilice xindy como motor de indexado. 
% * sanitize=none para no tener problemas con los acentos

\usepackage[acronym,nonumberlist, shortcuts,
xindy={language=spanish-traditional}, sanitize=none]{glossaries}
\makeglossaries

\usepackage{listings}
\lstloadlanguages{[LaTeX]TeX}
\lstset{
	language={[LaTeX]TeX},	%Lenguaje por defecto
	%
	% estilos
	keywordstyle=\bfseries\ttfamily\color[rgb]{.8,.1,.2},	% estilos de
% palabras clave, identificadores, etc...
	identifierstyle=\ttfamily,
	commentstyle=\color[rgb]{0.1,0.5,0.1},			 
	stringstyle=\ttfamily\color[rgb]{0.2,0.2,.7},			
	basicstyle=\footnotesize, 					
% the size of the fonts used for the code 
	%
	% numeraci\'on (desabilitadas en este momento :-)
	%numbers=left, 							
% where to put the line-numbers 
	numberstyle=\footnotesize, % size  of the fonts used for the
% line-numbers 
	stepnumber=1, 	% the step between two line-numbers. 
	numbersep=5pt, % how far the line-numbers are from the code 
	%
	% espacios
	showspaces=false, 						
% show spaces adding particular underscores 
	showstringspaces=false,	 	%underline spaces within strings 
	showtabs=false, 						
% show tabs within strings through particular underscores 
	tabsize=6,								
% sets default tab-size to 2 spaces
	%
	% cuadro
	backgroundcolor=\color{white}, 	% sets background color (needs package) 
	frame=single, 							
% adds a frame around the code
	rulecolor=\color[rgb]{.3,.3,.3},				
% set the frame's color. 
	captionpos=b, 							
% sets the caption-position to bottom 
	%
	% line breaking
	breaklines=true, % sets automatic line breaking 
	breakatwhitespace=false, % automatic breaks happen at whitespace 
	prebreak = \raisebox{0ex}[0ex][0ex]{\ensuremath{\hookleftarrow}}, % Nos
% dibuja una flecha ``guay'' cuando el c\'odigo no entra en una linea
	escapeinside=++,		% Para escapar a LaTeX. los acentos
}

\renewcommand{\acronymname}{Lista de Acr\'onimos}%
\addto\captionsspanish{%
\renewcommand*{\acronymname}{Lista de Acr\'onimos}%
}