\documentclass[11pt]{article}

\usepackage[spanish]{babel}   % (rec)  da soporte para castellano a LaTeX
\usepackage[utf8]{inputenc}    % (rec) soporte para tildes
\usepackage{url} %Para poder poner enlaces web
\RequirePackage{ifpdf}%para poder usar if pdf

%Para que haya enlaces internos en el documento
\ifpdf
    \usepackage[%
        pdftex,%
        colorlinks=true,% If false no colorlinks
        pdfborder={0 0 0},% No border around Links
        hyperindex,%
        plainpages=false,%
        bookmarksopen,%
        bookmarksnumbered,%
	breaklinks=true%
        ]{hyperref}
\else
    \usepackage[colorlinks=false,pdfborder={0 0 0}]{hyperref}
\fi

\usepackage{xcolor} %Para poder usar colores

%Cargo el paquete glossaries con diferentes opciones: 
% * acronym para crear listas de acr\'onimos
% * nonumberlist para que en la lista de acr\'onimos no ponga en qu\'e p\'aginas
% aparece  cada t\'ermino
% * shortcuts para poder utilizar los nombres cortos para acr\'onimos: \ac, etc
% * xindy para que utilice xindy como motor de indexado. 
% * sanitize=none para no tener problemas con los acentos

\usepackage[acronym,nonumberlist, shortcuts,
xindy={language=spanish-traditional}, sanitize=none]{glossaries}
\makeglossaries

\usepackage{listings}
\lstloadlanguages{[LaTeX]TeX}
\lstset{
	language={[LaTeX]TeX},	%Lenguaje por defecto
	%
	% estilos
	keywordstyle=\bfseries\ttfamily\color[rgb]{.8,.1,.2},	% estilos de
% palabras clave, identificadores, etc...
	identifierstyle=\ttfamily,
	commentstyle=\color[rgb]{0.1,0.5,0.1},			 
	stringstyle=\ttfamily\color[rgb]{0.2,0.2,.7},			
	basicstyle=\footnotesize, 					
% the size of the fonts used for the code 
	%
	% numeraci\'on (desabilitadas en este momento :-)
	%numbers=left, 							
% where to put the line-numbers 
	numberstyle=\footnotesize, % size  of the fonts used for the
% line-numbers 
	stepnumber=1, 	% the step between two line-numbers. 
	numbersep=5pt, % how far the line-numbers are from the code 
	%
	% espacios
	showspaces=false, 						
% show spaces adding particular underscores 
	showstringspaces=false,	 	%underline spaces within strings 
	showtabs=false, 						
% show tabs within strings through particular underscores 
	tabsize=6,								
% sets default tab-size to 2 spaces
	%
	% cuadro
	backgroundcolor=\color{white}, 	% sets background color (needs package) 
	frame=single, 							
% adds a frame around the code
	rulecolor=\color[rgb]{.3,.3,.3},				
% set the frame's color. 
	captionpos=b, 							
% sets the caption-position to bottom 
	%
	% line breaking
	breaklines=true, % sets automatic line breaking 
	breakatwhitespace=false, % automatic breaks happen at whitespace 
	prebreak = \raisebox{0ex}[0ex][0ex]{\ensuremath{\hookleftarrow}}, % Nos
% dibuja una flecha ``guay'' cuando el c\'odigo no entra en una linea
	escapeinside=++,		% Para escapar a LaTeX. los acentos
}

\renewcommand{\acronymname}{Lista de Acr\'onimos}%
\addto\captionsspanish{%
\renewcommand*{\acronymname}{Lista de Acr\'onimos}%
}
%\renewcommand{\listofacronymsname}{Lista de Acr\'onimos} % spanish title
%\listofacronyms
%\chapter*{Lista de acr\'onimos}
%\begin{acronym}


\newacronym{PT.x}{PT.x}{Paquete de trabajo con numeración \emph{x}}
\newacronym{T.x.y}{T.x.y}{Tarea con numeración \emph{y} a realizar dentro del
\acs{PT.x}}
\newacronym{CT}{CT}{Carga de trabajo}
\newacronym{UE.x}{UE.x}{Unidad entregable con numeración \emph{x}}
\newacronym{H.x}{H.x}{Hito de control con numeración \emph{x}}
\newacronym{Px}{Px}{Miembro \emph{x} del equipo de trabajo}
\newacronym{Mx}{Mx}{Mes relativo número \emph{x} (el proyecto comienza el mes
relativo 1)}
\newacronym{TER}{TER}{Tiempo estimado de realizaci\'on de la tarea}
\newacronym{Resp}{Resp.}{Persona responsable de la ejecuci\'on de la tarea}
\newacronym{Rev}{Rev.}{Persona responsable de la revisi\'on del trabajo
realizado}

\newacronym{CSE}{CSE}{Calidad en el Suministro Eléctrico}
\newacronym{PER}{PER}{Plan de Energ\'ias Renovables}
\newacronym{REE}{REE}{Red El\'ectrica de Espa\~na}
\newacronym{IEC}{IEC}{International Electrotechnical Commission}
\newacronym{PCC}{PCC}{Punto de Conexi\'on Com\'un}
\newacronym{GSC}{GSC}{Grupo de Se\~nal y Comunicaciones}
\newacronym{PFC}{PFC}{proyecto fin de carrera}
\newacronym{V}{V}{tensi\'on}
\newacronym{I}{I}{corriente}
\newacronym{W}{W}{velocidad de viento}
%\end{acronym}

\title{Creaci\'on de listas de acr\'onimos con \texttt{glossaries}}
\author{Digna Gonz\'alez Otero\\Itsas}

\begin{document}
\maketitle
\tableofcontents 

 \section{Introducci\'on}
 
 \subsection{Objetivo de documento}
El objetivo de este documento es explicar muy brevemente c\'omo funciona el
paquete \texttt{glossaries}, y en concreto c\'omo se puede utilizar para crear
listas de acr\'onimos. La documentaci\'on de este paquete se encuentra en CTAN,
y hay una versi\'on larga\footnote{\url{
http://mirror.ctan.org/macros/latex/contrib/glossaries/glossaries.pdf}}
y otra abreviada\footnote{\begin{scriptsize}\url{
http://mirror.ctan.org/macros/latex/contrib/glossaries/glossariesbegin.pdf}     
                                                                 
\end{scriptsize}}.

Se recomienda consultar estos manuales para informaci\'on m\'as detallada del
funcionamiento del paquete y para otros casos de aplicaci\'on.

\subsection{Alternativas}

Existen otras alternativas para crear listas de acr\'onimos, por ejemplo, el
paquete
\texttt{acronym}\footnote{\url{
http://mirror.ctan.org/macros/latex/contrib/acronym/acronym.pdf}}. Este
paquete es muy f\'acil de utilizar, pero es mucho menos flexible que
\texttt{glossaries}. Por ejemplo, no permite ordenar el listado de
acr\'onimos, y dispone de muchas menos opciones de configuraci\'on.

\section{Uso de \texttt{glossaries}}

\subsection{Principios b\'asicos}

Para usar \texttt{glossaries} en primer lugar tenemos que incluir el paquete
mediante \verb+\usepackage+, y despu\'es hacer que genere los archivos
intermedios necesarios con el comando \verb+\makeglossaries+. A la hora de
cargar el paquete podemos especificar algunas opciones de configuraci\'on.
Por ejemplo, podemos escribir lo siguiente en el
pre\'ambulo de nuestro documento:

\begin{lstlisting}
% Cargo el paquete glossaries con diferentes opciones: 
%  * acronym para crear listas de acr\'onimos
%  * nonumberlist para que en la lista de acr\'onimos no ponga 
% en qu\'e p\'aginas aparece  cada t\'ermino
% * shortcuts para poder utilizar los nombres cortos 
% para acr\'onimos: \ac, etc
% * xindy para que utilice xindy como motor de indexado. 
% * sanitize=none para no tener problemas con los acentos

 \usepackage[acronym,nonumberlist, shortcuts,%
xindy={language=spanish-traditional},%
sanitize=none]{glossaries}

\makeglossaries
\end{lstlisting}

Adem\'as, para que el t\'itulo de la lista de acr\'onimos aparezca en
castellano, tendremos que usar el siguiente comando:

\begin{lstlisting}
 \renewcommand{\acronymname}{Lista de Acr\'onimos}%
\addto\captionsspanish{%
\renewcommand*{\acronymname}{Lista de Acr\'onimos}%
}
\end{lstlisting}


Como se muestra en los comentarios del texto, las opciones cargadas son para lo
siguiente:

\begin{description}
 \item [acronym] indica que queremos crear listas de acr\'onimos.
 \item [nonumberlist] para que en la lista no indique junto a cada t\'ermino en
qu\'e p\'aginas aparece.
\item [shortcuts] para poder utilizar los comandos cortos para acr\'onimos. Por
ejemplo, \verb+\ac+, \verb+\acl+, etc.
\item [xindy] para utilizar \texttt{xindy} como motor de indexado. Para ello
hay que instalarlo en el equipo. En GNU/Linux, paquete \texttt{xindy}.
\item [sanitize=none] para evitar problemas con  los acentos al utilizar
algunos comandos, como \verb+\acs+ o \verb+\acl+.
\end{description}

\texttt{glossaries} necesita un motor de indexado para generar un \'indice de
las palabras y t\'erminos y luego poder ordenarlos. Por defecto usa
\texttt{makeindex}, pero en este ejemplo se ha utilizado \texttt{xindy} porque
dicen que es m\'as recomendable por su soporte para c\'odigo \LaTeX{} y su mayor
compatibilidad con otros idiomas (por ejemplo, por las tildes). 

A continuaci\'on, tambi\'en en el pre\'ambulo, hay que crear un listado de los
acr\'onimos que se quieran utilizar. Se puede crear el listado en un archivo a
parte e incluirlo con \texttt{input}.

El formato del listado de acr\'onimos es el siguiente:
\begin{verbatim}
 \newacronym{etiqueta}{nombrecorto}{nombre largo}
\end{verbatim}

A continuaci\'on se muestran algunos ejemplos:
\begin{lstlisting}
\newacronym{CSE}{CSE}{Calidad en el Suministro El\'ectrico}
\newacronym{PER}{PER}{Plan de Energ\'ias Renovables}
\newacronym{REE}{REE}{Red El\'ectrica de Espa\~na}
\newacronym{PCC}{PCC}{Punto de Conexi\'on Com\'un}
\newacronym{PFC}{PFC}{proyecto fin de carrera}
\end{lstlisting}

\subsection{Uso de los acr\'onimos}

A continuaci\'on, en el cuerpo del documento podemos referenciar los
acr\'onimos. Algunos de los comandos a utilizar son los siguientes:

\begin{description}
 \item [\textbackslash ac] muestra el acr\'onimo en su forma completa (nombre
largo + nombre corto entre par\'entesis) la primera vez que aparece el
acr\'onimo en el documento, y la forma corta el resto de veces.
 \item [\textbackslash acl] muestra la forma larga del acr\'onimo.
 \item [\textbackslash acs] muestra la forma corta.
 \item [\textbackslash acf] muestra la forma completa.
\end{description}

Por ejemplo, si ponemos \verb+\ac{PFC}+, la primera vez aparecer\'a la forma
completa, as\'i \ac{PFC}, pero las siguientes la forma corta, as\'i \ac{PFC}.

Adem\'as de estos comandos, hay otros para que la inicial empiece por
may\'uscula, por ejemplo cuando los usamos al principio de una frase. Son los
mismos comandos, pero con la inicial en may\'uscula. \verb+\Acl{PFC}+ genera
este resultado: \Acl{PFC}

\subsection{Listado de acr\'onimos}

Para generar el listado de acr\'onimos es necesario compilar con el sistema de
indexado. Para ello, basta con ejecutar el siguiente comando en el directorio
donde est\'a nuestro fichero:

\begin{lstlisting}
 makeglossaries NombreFichero
\end{lstlisting}

El orden de compilado es el siguiente:
\begin{enumerate}
 \item \texttt{latex} (o \texttt{pdflatex})
 \item \texttt{makeglossaries}
 \item \texttt{latex} (o \texttt{pdflatex})
 \item \texttt{latex} (o \texttt{pdflatex})
\end{enumerate}

Por \'ultimo, en el lugar del documento donde queramos generar el \'indice de
acr\'onimos tendremos que escribir \verb+\printglossaries+. Si queremos que
aparezca en el \'indice de contenidos, tendremos que utilizar el comando
\verb+\addcontentsline+

\begin{lstlisting}
\section*{Lista de acr\'onimos}
\printglossaries
\addcontentsline{toc}{section}{Lista de Acr\'onimos}
\end{lstlisting}




\newpage
\section{Resumen}
\begin{lstlisting}
 \documentclass{article}
 [...] % Aqu\'i se cargar\'ian m\'as paquetes
 \usepackage[acronym,nonumberlist, shortcuts,%
 xindy={language=spanish-traditional}, %
 sanitize=none]{glossaries}
 
 \renewcommand{\acronymname}{Lista de Acr\'onimos}%
\addto\captionsspanish{%
\renewcommand*{\acronymname}{Lista de Acr\'onimos}%
}
 
\makeglossaries
%\renewcommand{\listofacronymsname}{Lista de Acr\'onimos} % spanish title
%\listofacronyms
%\chapter*{Lista de acr\'onimos}
%\begin{acronym}


\newacronym{PT.x}{PT.x}{Paquete de trabajo con numeración \emph{x}}
\newacronym{T.x.y}{T.x.y}{Tarea con numeración \emph{y} a realizar dentro del
\acs{PT.x}}
\newacronym{CT}{CT}{Carga de trabajo}
\newacronym{UE.x}{UE.x}{Unidad entregable con numeración \emph{x}}
\newacronym{H.x}{H.x}{Hito de control con numeración \emph{x}}
\newacronym{Px}{Px}{Miembro \emph{x} del equipo de trabajo}
\newacronym{Mx}{Mx}{Mes relativo número \emph{x} (el proyecto comienza el mes
relativo 1)}
\newacronym{TER}{TER}{Tiempo estimado de realizaci\'on de la tarea}
\newacronym{Resp}{Resp.}{Persona responsable de la ejecuci\'on de la tarea}
\newacronym{Rev}{Rev.}{Persona responsable de la revisi\'on del trabajo
realizado}

\newacronym{CSE}{CSE}{Calidad en el Suministro Eléctrico}
\newacronym{PER}{PER}{Plan de Energ\'ias Renovables}
\newacronym{REE}{REE}{Red El\'ectrica de Espa\~na}
\newacronym{IEC}{IEC}{International Electrotechnical Commission}
\newacronym{PCC}{PCC}{Punto de Conexi\'on Com\'un}
\newacronym{GSC}{GSC}{Grupo de Se\~nal y Comunicaciones}
\newacronym{PFC}{PFC}{proyecto fin de carrera}
\newacronym{V}{V}{tensi\'on}
\newacronym{I}{I}{corriente}
\newacronym{W}{W}{velocidad de viento}
%\end{acronym}
\begin{document}
\section{Introducci\'on}

La \ac{CSE} es muy importante en los sistemas porque...

En este \ac{PFC}...

% Secci\'on de lista de acr\'onimos:
\printglossaries
\addcontentsline{toc}{section}{Lista de Acr\'onimos}
\end{document}
\end{lstlisting}

Contenido del fichero \texttt{acronyms.tex}:

\begin{lstlisting}[numbers=left]
\newacronym{CSE}{CSE}{Calidad en el Suministro El\'ectrico}
\newacronym{PER}{PER}{Plan de Energ\'ias Renovables}
\newacronym{REE}{REE}{Red El\'ectrica de Espa\~na}
\newacronym{PCC}{PCC}{Punto de Conexi\'on Com\'un}
\newacronym{PFC}{PFC}{proyecto fin de carrera}
\end{lstlisting}

Compilar seg\'un la siguiente secuencia:
\begin{enumerate}
 \item \texttt{latex} (o \texttt{pdflatex})
 \item \texttt{makeglossaries}
 \item \texttt{latex} (o \texttt{pdflatex})
 \item \texttt{latex} (o \texttt{pdflatex})
\end{enumerate}

\newpage
\printglossaries
\addcontentsline{toc}{section}{Lista de Acr\'onimos}
\end{document}
