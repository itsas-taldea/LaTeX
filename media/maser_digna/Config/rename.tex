%
% Este fichero contiene una lista de nombres (variables) internos
% de LaTeX, a los que puedes cambiar el nombre. Por ejemplo, puedes
% hacer que los cap�tulos se llamen "Secci�n" en vez de "Cap�tulo"
%
\renewcommand\bibname{Bibliograf\'{\i}a}        % as� el nombre de la secci�n Bibliograf�a ser� "Bibliograf�a".
\newcommand{\myname}{Digna González Otero}                 % nombre del autor.
%\newcommand{\myboss}{José Daniel Gutiérrez Porset}                 % nombre del supervisor.
\newcommand{\thesistitle}{\uppercase{Construcción e implantación de herramientas para la gestión colaborativa del conocimiento en el GSC}}        % t�tulo del trabajo.
\newcommand{\worktype}{Proyecto Fin de Carrera} % tipo de trabajo.
\newcommand{\upv}{Utils/ehu_logo.eps}          % fichero con el logo (p.e. para la portada).
\newcommand{\ibi}{Utils/logo_ingenieria.eps}          % fichero con el logo (p.e. para la portada).
\newcommand{\logoupv}{Utils/ehu_logo.png}          % fichero con el logo (p.e. para la portada).
\newcommand{\logoibi}{Utils/logo_ingenieria.png}          % fichero con el logo (p.e. para la portada).
%\renewcommand{\figurename}{xxx}                % nombre a pie de figura (xxx 1: bla-bla-bla).
%\renewcommand{\listfigurename}{yyy}            % nombre del �ndice de figuras.
\renewcommand{\tablename}{Tabla}
\renewcommand{\listtablename}{Índice de tablas}

    \makeatletter
    \def\thebibliography#1{\section*{REFERENCIAS\@mkboth
      {REFERENCIAS}{REFERENCIAS}}\list
      {[\arabic{enumi}]}{\settowidth\labelwidth{[#1]}\leftmargin\labelwidth
	\advance\leftmargin\labelsep
	\usecounter{enumi}}
	\def\newblock{\hskip .11em plus .33em minus .07em}
	\sloppy\clubpenalty4000\widowpenalty4000
	\sfcode`\.=1000\relax}
    \makeatother
