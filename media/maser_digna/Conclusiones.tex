\section{Conclusiones}
\label{conclusiones}

Un m\'aser es un \textbf{dispositivo que produce ondas electromagn\'eticas coherentes} mediante la amplificaci\'on por la emisi\'on estimulada de radiaci\'on. El t\'ermino se defini\'o como acr\'onimo de \textit{Microwave Amplification by Stimulated Emission of Radiation}, aunque es aplicable a un mayor rango de frecuencias, no s\'olo a las ondas de microondas.

El funcionamiento de los m\'aseres se basa en el fen\'omeno de la \textbf{emisi\'on estimulada de radiaci\'on}. En primer lugar se produce una inversi\'on de poblaci\'on de los \'atomos o mol\'eculas, es decir, se les excita para llevarlos a su nivel de energ\'ia superior. Despu\'es se hace incidir un fot\'on de una longitud de onda determinada, lo que produce que la mol\'ecula afectada baje a su nivel inicial de energ\'ia emitiendo dos fotones de la misma longitud de onda. Este fen\'omeno produce una reacci\'on en cadena que hace que se liberen r\'apidamente muchos fotones coherentes y con la misma longitud de onda.

Los m\'aseres producen \textbf{radiaci\'on coherente monocrom\'atica}, es decir, con una frecuencia muy estrecha, por lo que son adecuados como osciladores y como relojes at\'omicos. De hecho, algunos tipos de m\'aseres se usan como est\'andares temporales (como el m\'aser de hidr\'ogeno). Debido a su bajo nivel de ruido, los m\'aseres tambi\'en se utilizan como amplificadores en ciertos sistemas con requerimientos muy especiales, como en radiotelescopios.

Adem\'as de los m\'aseres creados por el hombre, ya exist\'ian en la naturaleza algunos naturales: son los \textbf{m\'aseres astron\'omicos}, que se producen en el espacio, especialmente en zonas donde se est\'an formando estrellas, y que permiten a los cient\'ificos obtener informaci\'on sobre el universo en que vivimos.
