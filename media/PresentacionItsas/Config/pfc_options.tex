%\mode<presentation>

\usetheme{Warsaw}
%\usetheme[titlepagelogo=ehu_logo,]{Digna}     
\usecolortheme[named=PineGreen]{structure}                       % Beamer Theme, Berlin, Warsaw, Ilmenau, Dresden
%\usecolortheme[RGB={109, 15, 36}]{structure} %Tirando a granate
%\usecolortheme[RGB={203, 76, 109}]{structure} %Rosa

\usecolortheme{freewilly}

% Establecemos una imagen como fondo de pantalla.
	\usebackgroundtemplate{
		% el \paperwidth y el \paperheigh son variables que devuelven el tamaño de cada diapositiva, así ajustamos la imagen a la diapositiva
		\includegraphics[width=\paperwidth,height=\paperheight]{Utils/fondo01.jpg}
	}
%\usecolortheme{seahorse}
%\usecolortheme{rose}                         % Beamer Color Theme

%Sin freewilly: Naranjas: Dandelion, Peach, Orange, Rojos: BitterSweet, **Mahogany, Maroon, Rosa: OrangeRed, Azul: Periwinkle

%Con freewilly: Naranjas: Peach, Rojos: BitterSweet, **Mahogany, ***Maroon, Rosa: OrangeRed, Azul: Periwinkle, CadetBlue, **MidnightBlue, NavyBlue, RoyalBlue, Blue, Verdes: **Green, ***ForestGreen, ****PineGreen, LimeGreen, ***OliveGreen, Marron: ****RawSienna, Sepia, **Brown, 
\usepackage{colortbl}
\usepackage[spanish]{babel}
\usepackage[utf8]{inputenc}
\usepackage[T1]{fontenc}
\usepackage{lmodern}
% Paquete que nos da el coloreado de sintaxis
\usepackage{listings}
% Indicamos a listing que debe cargar el dialecto LaTeX de TeX.
\lstloadlanguages{[LaTeX]TeX}
\usepackage{color}
\usepackage{xcolor}
% Paquete para poder incluir gráficos
\usepackage{graphicx}
	% Paquete para poder hacer dibujos (y también circuitos)
	%\usepackage[symbols]{circuitikz}
	\usepackage{tikz}

% Para poder usar \boldsymbol que nos permite poner negrita a expresiones ``matemáticas'' en modo matemático.
\usepackage{amsmath}


% Establecemos los valores por defecto de Listings
\lstset{
	language={[LaTeX]TeX},						% Lenguaje por defecto
	%
	% estilos
	keywordstyle=\bfseries\ttfamily\color[rgb]{.8,.1,.2},	% estilos de palabras clave, identificadores, etc...
	identifierstyle=\ttfamily,
	commentstyle=\color[rgb]{0.1,0.5,0.1},			 
	stringstyle=\ttfamily\color[rgb]{0.2,0.2,.7},			
	basicstyle=\footnotesize, 						% the size of the fonts used for the code 
	%
	% numeración (desabilitadas en este momento :-)
	%numbers=left, 								% where to put the line-numbers 
	numberstyle=\footnotesize, 					% size of the fonts used for the line-numbers 
	stepnumber=1, 							% the step between two line-numbers. 
	numbersep=5pt, 							% how far the line-numbers are from the code 
	%
	% espacios
	showspaces=false, 							% show spaces adding particular underscores 
	showstringspaces=false,	 					% underline spaces within strings 
	showtabs=false, 							% show tabs within strings through particular underscores 
	tabsize=6,									% sets default tab-size to 2 spaces
	%
	% cuadro
	backgroundcolor=\color{white}, 				% sets background color (needs package) 
	frame=single, 								% adds a frame around the code
	rulecolor=\color[rgb]{.3,.3,.3},					% set the frame's color. 
	captionpos=b, 								% sets the caption-position to bottom 
	%
	% line breaking
	breaklines=true, 							% sets automatic line breaking 
	breakatwhitespace=false, 					% automatic breaks happen at whitespace 
	prebreak = \raisebox{0ex}[0ex][0ex]{\ensuremath{\hookleftarrow}} % Nos dibuja una flecha ``guay'' cuando el código no entra en una linea
}


% Nos definimos unas funciones para definir los colores

% Nos define una función \red con un parámetro que será sustituido en #1
\newcommand{\red}[1]{\color[rgb]{1,0,0} #1}

\newcommand{\black}[1]{\color[rgb]{0,0,0} #1}
\newcommand{\gray}[1]{\color[rgb]{.7,.7,.7} #1}
\newcommand{\lightblue}[1]{\color[rgb]{.2,.5,1} #1}
\newcommand{\mygreen}[1]{\color[rgb]{.2,.7,.2} #1}

% Nota, \color es una función del paquete xcolor, que beamer carga por defecto.

\title{Presentación del grupo Itsas}
%\subtitle{Grupo de Software Libre de la UPV/EHU}
\author[Digna Gonz\'alez Otero]{Digna Mar\'ia González Otero}
%\institute{Itsas \\[0.5cm] digna.gonzalez@gmail.com}
\date{}


\newenvironment{specialframe}{\begin{frame}[fragile,environment=specialframe]}{\end{frame}}

\newenvironment{myplainframe}{\begin{frame}[fragile,plain,environment=specialframe]}{\end{myplainframe}}
      

